\chapter{Results}
\label{chap:results}
\section{Evaluation results}
The evaluation has been made for the \ac{LSB} (the reference), the \ac{RLSB} and the \ac{SSW} technique with the \texttt{Amplify}, \texttt{Convolution} (with a low-pass filter) and \texttt{FFTAmplify} degradations. The scores (number of false bits) are :

\begin{figure}[h!]
\centering
\begin{tabular}{|c|c|c|c|}
\hline
\backslashbox{Algorithm}{Degradation} & \texttt{Amplify} & \texttt{Convolution} & \texttt{FFTAmplify} \\
\hline
\ac{LSB} & 5 & 10 & 7 \\
\hline
\ac{RLSB} & 31 & 37 & 7\\
\hline
\ac{SSW} & 52 & 71 & 52\\
\hline
\end{tabular}
\end{figure}

Something important to consider is that what we thought was the most resistant (SSW) is actually the less resistant, certainly because our implementation is not perfect.

More surprisingly, our custom \ac{RLSB} algorithm doesn't work quite well. However, this is certainly because if the bit to encode is $1$, and the encoding is on five bits, the last bits are $11111$. Only doing $11111 + 1$ would ruin the bit.
\section{Auditory results}
The auditory evaluation, made on about ten persons, wasn't really conclusive because of the lack of precision of the answers.

The results are shown on figure \ref{fig:auditory}.
\begin{figure}[h!]
\centering
\begin{tabular}{|c|c|}
\hline
File & Satisfaction \\
\hline
Brassens & \\
\hline
LSB1 & ++ \\
LSB3 & ++ \\
LSB5 & ++ \\
RLSB1 & ++ \\
RLSB3 & ++ \\
RLSB5 & ++ \\
SSW10 & -- \\
SSW20 & -- \\
SSW100 & -- \\
\hline
Solo  & \\
\hline
LSB1 & ++ \\
LSB3 & ++ \\
LSB5 & ++\\
RLSB1 & ++\\
RLSB3 & ++\\
RSLB5 & ++\\
SSW10 & -\\
SSW20 & --\\
SSW100 & --\\
\hline
\end{tabular}
\caption{Auditory results}
\label{fig:auditory}
\end{figure}

A high satisfaction indicates that the noise due to the watermarking couldn't be heard.

For the \ac{LSB} algorithms, the number to the right of the file is the number of bits used.

For the \ac{SSW} algorithm, it is the amplitude used for the SSW.

We could only note that the \ac{SSW} algorithm is more suited to songs with a very rich harmonic content.