\chapter{Watermarking}
\section{Presentation}
\section{Existing work}
\section{Techniques}
\subsection{Least Significant Bit}
\subsection{Spread-spectrum Watermarking}

The purpose of the \ac{SSW} method is to embed the hidden data of the watermark in the frequency domain of the host signal. Hence, \ac{SSW} algorithm relies on a time-to-frequency transform and on its inverse transform. In fact, any frequency transform can be used with this method, as long as the same transform is used for watermarking insertion and detection. For example, in this project, we used the Discrete Fourier Transform.

~

The core of this method is based on a random sequence of N numbers, whose values are either $1$ or $-1$, with a equal distribution for both. This sequence is the key of decryption and thus must be known by the emitter and the receiver.

~

Therefore, the first step is to generate this sequence.

\subsubsection{Watermark insertion}

In order to embed data with the \ac{SSW} method, the host signal must firstly be divided into chunks of given size (for example, chunks of $512$ samples). The \ac{SSW} algorithm can then be applied to each of these chunks in order to embed the hidden data. Only 1 bit of hidden data can be embedded in a chunk.

~

On each chunk, the frequency transform is then applied. The result is an array of half the size of the chunks, each cell of it containing the energy and phase for a certain frequency bin of this chunk.

~

The next step is to select N (the size of the random sequence) frequency bins that are going to be modified. Therefore, N has to be less than half the chunk size. The goal is to avoid inaudible portions of the frequency spectrum because they are more likely to create audible noise when modified. The best is to keep frequency bins in the $200$ Hz - $2$ kHz subband.

~

Let's call $F$ the sequence of the magnitudes (in decibels) of the selected frequency bins and $S$ the random sequence. The following computation is then performed in order to insert 1 bit of hidden information into 1 chunk:

~

$F[i] = F[i] + W[k] \times \delta \times S[i]$, for $i$ varying from 1 to N.

~

\noindent where $W$ is the hidden data to be embedded ($W[k]$ is equal to 1 for bit 1 and -1 for bit 0), $k$ the current bit of hidden data and $\delta$ the amplitude of the watermark (usually between $0.5$ and $2.5$ dB).

~

After the modification on the frequency magnitudes, the time signal is reconstructed using the original phases and the inverse frequency transform.

\subsubsection{Watermark detection}

Just like in the embedding part, the signal must be divided into chunks. They have to be the same size as during embedding.

~

The frequency transform is also applied and the same frequency bins as before are also selected. Magnitudes in decibels are computed as well but instead of modifying them, a correlation is calculated, just as follows :

~

$C = F \cdot \delta S$

~

\noindent where $F$ is the sequence of selected frequencies magnitudes, $S$ the random sequence and $\cdot$ the normalized dot product.

~

When the chunk has not been watermarked, $C$ should be close to $0$ because $F$ would be close to a random sequence, and when the chunk is watermarked, $C$ should be closer to $1$ because $F$ would be equal to $random + \delta S$ and $\delta S \cdot \delta S = 1$. When the bit 0 is embedded, $F$ is equal to $random - \delta S$, so $C$ should be closer to $-1$.

~

Hence, for each chunk, it is possible to determine whether or not there is a watermark (by using a threshold) and if so, which bit of data is embedded. Doing it for each chunk allows for the reconstruction of the entire hidden data.

~

A bigger $N$ would allow better correlation values but also a more audible watermark. The same rule applies to $\delta$.

~

There are several methods that can improve watermark detection, like cepstrum filtering, but these were not implemented in this project.

\subsection{Compression-Expansion}\cite{foo2010}

Compression-expansion is a time-based method for watermarking, unlike S.S.W. which is a spectral-based method.\\
Its purpose is based on dividing a signal into segments of frames that overlap each other, calculate a local mask and taking specific coefficients according to the mask previously calculated from a frame to add them to the end of the following one. The technique is detailed in this subsection.

\paragraph{Purpose of the compression-expansion of segments}
Let's consider a signal. One divide it into segments that are containing the same number of frames.\\
The division must be in respect of the following purpose : each segment half-overlaps the following one. A Hanning window is then applied to the segments, as showed by the following figure :
\begin{figure}[H]
\center{\includegraphics{images/Image1.png}}
\caption{\label{frames} Division into frames of a signal}
\end{figure}
For each bit of the watermark, two segments are used. So that no confusion is made, the third segment that follows the two segments used is left.\\
A discret cosine transform (D.C.T.) is applied on the two segments that are under consideration. Here is used a local auditory mask, which is calculated thanks to a psychoacoustic model. We will see later how this mask is calculated.

According to the mask, in the first segment, one takes the 8 first coefficients of the D.C.T. that are greater than the mask. One must look for those coefficients from the end of the segment. As one has removed coefficients from a segment, it is compressed\\
Those coefficients are then added to the beginning of the following segment. It is thus expanded.\\
The following figure sums up this operation :
\begin{figure}[H]
\center{\includegraphics{images/Image2.png}}
\caption{\label{compression-expansion} Compression and expansion of the segments}
\end{figure}

An inverse D.C.T. is applied then to come back to the time domain. The differences between the original signal and the reconstructed one yields a signal made up of little waveforms that have the shape of a diamond, such as this figure shows :
\begin{figure}[H]
\center{\includegraphics{images/Image3.png}}
\caption{\label{diamond} Diamond obtained by the difference of original and reconstructed waveforms}
\end{figure}

\paragraph{Use of the diamonds}
Depending on the original signal, either mono, either stereo, the method differs. In each case, the diamonds that are previously obtained by the difference of the original signal and the reconstructed one are used, but in different ways.

Let's consider a mono signal. The principle is here to obtain a diamond to embed the bit "1" (therefore, apply the method described in the previous paragraph), and nothing to embed the bit "0" (no calculation).\\
As it is not sure that the processing of this type of watermarking begins exactly at the beginning of the signal, one must know where's the first bit embedded. One considers then that, when the encoding begins, a diamond is calculated.

For a stereo signal, one uses the two channels. On the left channel, a diamond is calculated to embed "1", and the other way round for "0" (on the right channel). The two channels enables the fact that there can be blanks on both channels at the same time, therefore having different time intervals between the diamonds.
\begin{figure}[H]
\center{\includegraphics{images/Image5.png}}
\caption{\label{stereo watermarking} Series of diamonds for a stereo signal}
\end{figure}

The next step is to detect those diamonds. The main problem generated by the compression-expansion is that, in some cases, the distortion can be difficult to detect (the diamonds are to tiny), as this is showed in this figure :
\begin{figure}[H]
\center{\includegraphics{images/Image4.png}}
\caption{\label{distortion} Example of a to much tiny diamond}
\end{figure}

Such a problem can be easily avoided in stereo signals, because the watermarks can be separated with blanks on both channels. But on a mono signal, if the diamond is considered as noise (therefore, not taken into consideration), it will yield a false detection of the watermark. This is a constraint for mono signals : some types of signals (such as those which encode speeches) can't be used to ensure a good accuracy. These types of signals have often parts where the energy is low, which means a low amplitude. As the diamond is obtained by taking the difference of the original signal and the reconstructed one, this difference is as low as the amplitude of the signal is weak.

To detect the watermark inside a signal, the original non-watermarked signal is needed to calculate the diamonds. The detection of the diamonds can be done with several methods : some are described here.\\
\begin{itemize}
\item \textbf{Sum of absolute difference} : one computes the sums of absolute differences between the watermarked signals and the original signals. So as to detect a diamond, which duration is equal to 2 segments of frames that overlap each other by 50\% in the signal, in addition to one frame unused due to the 50\% overlap, each sum is calculated on the base of 3 segments as described previously. On the one hand, when there is a diamond (thus, a distortion on the watermarked signal), the sum of the absolute values of the differences during the distortion is found to be greater than 5,5. On the other hand, the sum of the absolute values of the differences is always under 0,8. Thus, a threshold of $\frac{5,5 + 0,8}{2} = 3,15$ can be taken to detect a diamond.\\
This method is very simple and very accurate when no attack is done on the watermarked signal. But when it is damaged, several errors of detection can be made.
\item \textbf{Calculation of gradients} : a diamond has two slopes that make the shape of a triangle, such as showed on the figure below :
\begin{figure}[H]
\center{\includegraphics[scale=0.7]{images/Image6.png}}
\caption{\label{slopes} The slopes for a diamond}
\end{figure}
The derivative of each slope should be a constant non-zero function. It is experimentally found that the constant values of these function are greater than 1.7 for the left slope and lower than -1.5 for the right slope. These criteria can be used to ensure that the segments considered where used to calculate a diamond or not.\\
From the experience, this type of searching works well on attacked signals with noise and re-sampling. Nevertheless, the results are less satisfying when the watermarked signal is compressed using a low-pass filter or M.P. 3 algorithm.
\item \textbf{Cross-correlation with reference triangle} : Experimental results show that the amplitudes of the diamonds using different types of audio signals are quite similar. Therefore, for a segment of 512 samples, the length of the envelope is the same as the one of the diamond, which is 768 samples ($\frac{3}{2}$ times the segment length). The average amplitude is 0,06 in the middle of the envelope (if the frame size changes, then these values are no more applicable).
\begin{figure}[H]
\center{\includegraphics[scale=0.7]{images/Image7.png}}
\caption{\label{envelope} Envelope for a watermark extraction}
\end{figure}
The cross-correlation between this shape and the signal with the diamonds is calculated. If the value is near 1, it means there is a diamond, whereas if it is near 0, there are non-distorted frames.\\
Experiments indicate that a cross-correlation value over 0,78 is a diamond and under 0,31 represents non-distorted frames. If the value is between 0,78 and 0,31, then another detection method is applied. This method gives higher detection accuracy than the previous one.
\end{itemize}

\paragraph{Calculation of the mask}
The signal is segmented into overlapped frames. Then, the short time Fourier transform is applied in order to get a spectral representation. However, another scale is used : the Bark Scale. The concept of the Bark scale is based on a psychoacoustic phenomenon : the basilar membrane in the hearing mechanism analyses the incoming sound through a spatial-spectral analysis. This is done in small regions of the basilar membrane called “critical bands”.\\
The \textbf{Bark scale} is obtained from the frequency range with the following formula :
$$z_i = 13\textnormal{tan}^{-1}\frac{0,76f_i}{1000}+3,5\textnormal{tan}^{-1}\left(\frac{f_i}{7500}\right)^2$$
In the following graph is given the spectral representation of the 7\textsuperscript{th} segment of the file \textbf{\textit{input\_mono.wav}}, given in the folder \textbf{\textit{output}}. This was calculated with \textbf{\textit{MatLab}} :
\begin{figure}[H]
\center{\includegraphics[scale=0.5]{images/Image8.png}}
\caption{Spectral representation of the 7\textsuperscript{th} segment of \textbf{\textit{input\_mono.wav}}}
\end{figure}

Then, one needs to calculate the power spectra. The formula is given below :
$$S_p(j\omega) = 	|\mathcal{F}(s(t)w(t))|$$
where $sw(t) = s(t)w(t)$ is the windowed segment, and $\mathcal{F}$ is the Fourier transform (which explains the transformation from $t$ to $j\omega$).

For each critical band of the basilar membrane, one computes its energy as follows :
$$S_{pz}(z) = \sum_{\omega = LBZ}^{\omega = HBZ}S_p(j\omega)$$
where $LBZ$ and $HBZ$ are respectively the lower and higher frequencies of the critical band. z goes from 1 to the total of critical bands, and one does the sum to get the total energy.\\
In the graph below is represented in red the energy per critical band :
\begin{figure}[H]
\center{\includegraphics[scale=0.5]{images/Image9.png}}
\caption{Energy per critical band and spectral representation}
\end{figure}

At this step, one needs also to compute the basilar membrane spreading function. This function determines how much of the energy of each 
critical band is contributed to the neighbouring bands.\\
This function in dB is :
$$B_k = 15,91+7,5(k + 0,474) - 17,5\sqrt{1 + (k + 0,474)^2}$$
k takes its values in $\mathbb{Z}$.
\begin{figure}[H]
\center{\includegraphics[scale=0.5]{images/Image10.png}}
\caption{Representation of $B$ in function of $z$}
\end{figure}
The spreading across bands is computed by the convolution of the previous function and the energy per critical band :
$$S_m(z) = S_pz(z)*B(z)$$
\begin{figure}[H]
\center{\includegraphics[scale=0.5]{images/Image11.png}}
\caption{Representation of $S_m$ as a function of z (red curve)}
\end{figure}

With the different things computed, one can now calculate the masking threshold.\\
First, one must get the spectral flatness measure :
$$SFM = 10\textnormal{log}\left(\frac{\prod_{z = 1}^{z = Z_t} S_{pz}(z)}{\frac{1}{Z_t}\sum_{z = 1}^{z = Z_t} S_{pz}(z)}\right)^\frac{1}{Z_t}$$

One can now have the tonality factor $\alpha$ :
$$\alpha = \textnormal{min}\left(\frac{SFM}{-60} ; 1\right)$$

The masking threshold is given by :
$$T_{norm}(z) = \frac{10^{\textnormal{log} S_m(z)-\frac{O(z)}{10}}}{P_z}$$
where $O(z)$ is the masking energy offset :
$$O(z) = \alpha(14,5 + z) + 5,5(1 - \alpha)$$
and $P_z$ is the number of points in each band $z$, $z \in [\![1 ; Z_t]\!]$.

\paragraph{Results of the method}
Experiments have been made on three types of signals :
\begin{itemize}
\item rock music
\item classical music
\item male speech
\end{itemize}
Rock music have a very high signal energy, unlike classical music. A speech has also a moderate signal energy, but sequences are very energied-low.\\
All signals are sampled at 44,1 kHz and quantized with 16 b. Files are either mono (segment size : 512), either stereo (segment sizes : 256, 512 or 1024).

The information was a binary $64\times64$ image.

The audio quality of the watermarked signal was measured with several methods. Listeners evaluated the quality through comparing both original and modified signals. And to get more objective results, the signal to noise ratio (S.N.R.) was used. This is given by the formula :
$$SNR = 10\textnormal{log}\frac{\sum_{n} x(n)^2}{\sum_{n} (x(n) - y(n))^2}$$
where $x$ is the original signal and $y$ the watermarked one.\\
This experiment leads to conclude that the watermarked signals are not damaged perceptually. Furthermore, the S.N.R. values obtained for watermarked signals are around 40 dB. Notice that, for stereo signals, the S.N.R. is calculated on each channel.\\
The requirement of the I.F.P.I. (International Federation of the Phonographic Industry) is 20 dB. Therefore, it can be concluded that the watermarks do not affect the signal perceptually speaking.

To test the robustness, several attacks were made. The robustness is calculated with the bit error rate, which is the number of false bits recovered divided by the total number of bits.\\
The attacks were :
\begin{itemize}
\item Attack free
\item Noise addition (decrease of the S.N.R. of 20 dB)
\item Re-sampling (down-sampling to 22,05 kHz and up-sampling to 44,1 kHz then)
\item Re-quantization (down-quantized to 8 b and up-quantized to 16 b again)
\item Echo adding (delay : 0,001 s)
\item Low-pass filtering (cutoff frequency at 6 kHz)
\item M.P. 3 compression (32 kb/s)
\item Amplitude variation (increased of 30 \% and decreased then by 20 \%)
\item Time-scale modification (lengthened by 10 \% and decreased then by 5 \%)
\item Pitch scaling
\end{itemize}

From the results obtained, stereo signals are better to enclose a watermark. Mono signals are easier to be attacked, and the information recovered is false. The same observation is made when the segments have too low frames. It is because the maximum amplitude of the diamonds are lower as the segments are smaller.\\
Compressions, which are generally damaging, are here well supported.