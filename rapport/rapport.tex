\documentclass[a4paper,11pt,oneside]{report}

%\date{September the 15th, 2013}
\usepackage{geometry}
\usepackage[utf8]{inputenc}
\usepackage[english]{babel}
\usepackage[T1]{fontenc}
\usepackage{relsize}
\usepackage{color}
\definecolor{dkgreen}{rgb}{0,0.6,0}
\definecolor{gray}{rgb}{0.5,0.5,0.5}
\definecolor{mauve}{rgb}{0.58,0,0.82}

\usepackage{listings}
\usepackage{float}
\usepackage{kpfonts}
\usepackage{verbatimbox}
\usepackage{datetime}
% more figures per page
\renewcommand\floatpagefraction{.9}
\renewcommand\topfraction{.9}
\renewcommand\bottomfraction{.9}
\renewcommand\textfraction{.1}   
\setcounter{totalnumber}{50}
\setcounter{topnumber}{50}
\setcounter{bottomnumber}{50}

\usepackage{graphicx}
%% \newcommand{\foofoo}{\hspace{-2.3pt}$\bullet$ \hspace{5pt}}


\lstset{ %
language=C++,                % choose the language of the code
basicstyle=\footnotesize,       % the size of the fonts that are used for the code
numbers=left,                   % where to put the line-numbers
numberstyle=\footnotesize,      % the size of the fonts that are used for the line-numbers
stepnumber=1,                   % the step between two line-numbers. If it is 1 each line will be numbered
numbersep=5pt,                  % how far the line-numbers are from the code
backgroundcolor=\color{white},  % choose the background color. You must add \usepackage{color}
showspaces=false,               % show spaces adding particular underscores
showstringspaces=false,         % underline spaces within strings
showtabs=false,                 % show tabs within strings adding particular underscores
frame=single,           % adds a frame around the code
tabsize=2,          % sets default tabsize to 2 spaces
captionpos=b,           % sets the caption-position to bottom
breaklines=true,        % sets automatic line breaking
breakatwhitespace=false,    % sets if automatic breaks should only happen at whitespace
escapeinside={\%*}{*)}          % if you want to add a comment within your code
}


\lstset{
  literate={ù}{{\`u}}1
           {é}{{\'e}}1
           {è}{{\'e}}1
           {à}{{\`a}}1
}


\geometry{margin=2cm}
\geometry{headheight=15pt}
\usepackage{hyperref}
\usepackage{fancyhdr}
\usepackage[xindy,toc]{glossaries}
\usepackage{natbib}
\usepackage{fancyvrb}
\usepackage{float}
\usepackage{algorithm2e}
\usepackage{CJKutf8}
\usepackage[footnote,smaller]{acronym}
\usepackage[perpage]{footmisc}
\usepackage[]{titlesec}
\makeatletter
\def\ttl@mkchap@i#1#2#3#4#5#6#7{%
    \ttl@assign\@tempskipa#3\relax\beforetitleunit
    \vspace{\@tempskipa}%<<<<<< REMOVE THE * AFTER \vspace
    \global\@afterindenttrue
    \ifcase#5 \global\@afterindentfalse\fi
    \ttl@assign\@tempskipb#4\relax\aftertitleunit
    \ttl@topmode{\@tempskipb}{%
        \ttl@select{#6}{#1}{#2}{#7}}%
    \ttl@finmarks  % Outside the box!
    \@ifundefined{ttlp@#6}{}{\ttlp@write{#6}}}
\makeatother

\renewcommand*{\acsfont}[1]{\brand{#1}}

\makeglossaries

\pagestyle{fancy}


\newcommand{\brand}[1]{\textsc{\textbf{#1}}}
\let\oldpar\paragraph
\renewcommand{\paragraph}[1]{\oldpar{#1}\mbox{}\\}


\DeclareMathOperator{\power}{power}
\DeclareMathOperator{\phase}{phase}
\DeclareMathOperator{\recompute}{recompute}

\rhead{Audio watermarking}
\begin{document}
\selectlanguage{english}
\begin{titlepage}
  \begin{center}

    \textsc{\LARGE Audio Watermarking}\\[1cm]
    
  \end{center}
  
  \vspace*{\stretch{2}}
  \begin{flushbottom}
   \begin{flushleft}
    \underline{Students} : Jean-Michaël \textsc{Celerier}, Abdelhamid \textsc{Cherif}, Alban \textsc{De Martin}, Quentin \textsc{Midy}, Marc \textsc{Queiros Soares}, Valentin \textsc{Sarthou} - \today \\
   \end{flushleft}
  \end{flushbottom}
\end{titlepage}
\clearpage
\tableofcontents
\newglossaryentry{watermarking}
{
  name=watermarking,
  description={action of inserting a watermark into some data}
}

\chapter*{Introduction}
In the fields of digital technologies, it is possible to consider an information as a series of bits which represents a message. Thus, watermarking is the technical solution which consists in hiding a message in a given signal. 

This project will deal with the main algorithms used to watermark data in audio signals.

We will first present the theory behind common algorithms, present the evaluation methods, and then move on to the project and implementation parts : how the project was managed, and how the project was implemented.
\chapter{Watermarking}
\Gls{watermarking} is cool! Le ssw aussi\cite{cox1997secure}
\section{Presentation}
\section{Existing work}
\section{Techniques}
\subsection{Least Significant Bit}
\subsection{Spread-spectrum Watermarking}

The purpose of the spread-spectrum watermarking (SSW) method is to embed the hidden data of the watermark in the frequency domain of the host signal. Hence, SSW algorithm relies on a time-to-frequency transform and on its inverse transform. In fact, any frequency transform can be used with this method, as long as the same transform is used for watermarking insertion and detection. For example, in this project, we used the Discrete Fourier Transform.

~

The core of this method is based on a random sequence of N numbers, whose values are either $1$ or $-1$, with a equal distribution for both. This sequence is the key of decryption and thus must be known by the emitter and the receiver.

~

Therefore, the first step is to generate this sequence.

\subsubsection{Watermark insertion}

In order to embed data with the SSW method, the host signal must firstly be divided into chunks of given size (for example, chunks of $512$ samples). The SSW algorithm can then be applied to each of these chunks in order to embed the hidden data. Only 1 bit of hidden data can be embedded in a chunk.

~

On each chunk, the frequency transform is then applied. The result is an array of half the size of the chunks, each cell of it containing the energy and phase for a certain frequency bin of this chunk.

~

The next step is to select N (the size of the random sequence) frequency bins that are going to be modified. Therefore, N has to be less than half the chunk size. The goal is to avoid inaudible portions of the frequency spectrum because they are more likely to create audible noise when modified. The best is to keep frequency bins in the $200$ Hz - $2$ kHz subband.

~

Let's call $F$ the sequence of the magnitudes (in decibels) of the selected frequency bins and $S$ the random sequence. The following is then performed :

~

$F[i] = F[i] + W[k] \times \delta \times S[i]$, for $i$ varying from 1 to N.

~

\noindent where $W$ is the hidden data to be embedded ($W[k]$ is equal to 1 for bit 1 and -1 for bit 0), $k$ the current bit of hidden data and $\delta$ the amplitude of the watermark (usually between $0.5$ and $2.5$ dB).

~

After the modification on the frequency magnitudes, the time signal is reconstructed using the original phases and the inverse frequency transform.

\subsubsection{Watermark detection}

Just like in the embedding part, the signal must be divided into chunks. They have to be the same size as during embedding.

~

The frequency transform is also applied and the same frequency bins as before are also selected. Magnitudes in decibels are computed as well but instead of modifying them, a correlation is calculated, just as follows :

~

$C = F \cdot \delta S$

~

\noindent where $F$ is the sequence of selected frequencies magnitudes, $S$ the random sequence and $\cdot$ the normalized dot product.

~

When the chunk has not been watermarked, $C$ should be close to $0$ because $F$ would be close to a random sequence, and when the chunk is watermarked, $C$ should be closer to $1$ because $F$ would be equal to $random + \delta S$ and $\delta S \cdot \delta S = 1$. When the bit 0 is embedded, $F$ is equal to $random - \delta S$, so $C$ should be closer to $-1$.

~

Hence, for each chunk, it is possible to determine whether or not there is a watermark (by using a threshold) and if so, which bit of data is embedded. Doing it for each chunk allows for the reconstruction of the entire hidden data.

~

A bigger $N$ would allow better correlation values but also a more audible watermark. The same rule applies to $\delta$.

~

There are several methods that can improve watermark detection, like cepstrum filtering, but these were not implemented in this project.

\subsection{Compression-Expansion}

Compression-expansion is a time-based method for watermarking, unlike S.S.W. which is a spectral-based method.\\
Its purpose is based on dividing a signal into segments of frames that overlap each other, calculate a local mask and taking specific coefficients according to the mask previously calculated from a frame to add them to the end of the following one. The technique is detailed in this subsection.

\paragraph{Purpose of the compression-expansion of segments}
Let's consider a signal. One divide it into segments that are containing the same number of frames.\\
The division must be in respect of the following purpose : each segment half-overlaps the following one. A Hanning window is then applied to the segments, as showed by the following figure :
\begin{figure}[H]
\center{\includegraphics{images/Image1.png}}
\caption{\label{frames} Division into frames of a signal}
\end{figure}
For each bit of the watermark, two segments are used. So that no confusion is made, the third segment that follows the two segments used is left.\\
A discret cosine transform (D.C.T.) is applied on the two segments that are under consideration. Here is used a local auditory mask, which is calculated thanks to a psychoacoustic model. We will see later how this mask is calculated.

According to the mask, in the first segment, one takes the 8 first coefficients of the D.C.T. that are greater than the mask. One must look for those coefficients from the end of the segment. As one has removed coefficients from a segment, it is compressed\\
Those coefficients are then added to the beginning of the following segment. It is thus expanded.\\
The following figure sums up this operation :
\begin{figure}[H]
\center{\includegraphics{images/Image2.png}}
\caption{\label{compression-expansion} Compression and expansion of the segments}
\end{figure}

An inverse D.C.T. is applied then to come back to the time domain. The differences between the original signal and the reconstructed one yields a signal made up of little waveforms that have the shape of a diamond, such as this figure shows :
\begin{figure}[H]
\center{\includegraphics{images/Image3.png}}
\caption{\label{diamond} Diamond obtained by the difference of original and reconstructed waveforms}
\end{figure}

\paragraph{Use of the diamonds}
Depending on the original signal, either mono, either stereo, the method differs. In each case, the diamonds that are previously obtained by the difference of the original signal and the reconstructed one are used, but in different ways.

Let's consider a mono signal. The principle is here to obtain a diamond to embed the bit "1" (therefore, apply the method described in the previous paragraph), and nothing to embed the bit "0" (no calculation).\\
As it is not sure that the processing of this type of watermarking begins exactly at the beginning of the signal, one must know where's the first bit embedded. One considers then that, when the encoding begins, a diamond is calculated.

For a stereo signal, one uses the two channels. On the left channel, a diamond is calculated to embed "1", and the other way round for "0" (on the right channel). The two channels enables the fact that there can be blanks on both channels at the same time, therefore having different time intervals between the diamonds.

The next step is to detect those diamonds. The main problem generated by the compression-expansion is that, in some cases, the distortion can be difficult to detect (the diamonds are to tiny), as this is showed in this figure :
\begin{figure}
\center{\includegraphics{images/Image4.png}}
\caption{\label{distortion} Example of a to much tiny diamond}
\end{figure}


\chapter{Evaluation}
\section{Main evaluation idea}
To evaluate the robustness of the watermark techniques that we implemented, we decided to compare the strength of the techniques against different attacks with the \ac{LSB} technique. Since the \ac{LSB} technique was the most easy one to implement, this has been decided from the begining. Hence, we compute the score of the other techniques with the score of the \ac{LSB} algorithm as a reference. \\

The scores are computed by calculating the number of false bits in a watermarked data after altering it with a degradation. To do so, we implemented different benchmark algorithms that degrade the signal. \\
\section{Existing work}
Many evaluation tools for watermarking techniques exist and they have inspired our work. 
The first of them, which is the one that inspired most of our evaluations, is the \textbf{StirMark Benchmark}. It's a generic tool for simple robustness testing of image watermarking algorithms. 

Even if we focused on the audio watermark, it was helpful to learn about it. It introduces several geometric distortions to de-synchronise watermarking algorithms such as:
\begin{itemize}
 \item Add a noise to the signal
 \item Amplify  the signal
 \item High pass / Low pass filters
 \item Delay...
\end{itemize}

An audio version of the \brand{StirMark Benchmark}, called \brand{StirMark Benchmark for Audio} used to exist, however, it was removed from the internet and is now only accessible using \brand{Google Cache}. This was the main basis for our work.

Another existing evaluation tool is the \textbf{Watermark Evaluation TestBed (WET)}. However, we did not had the time to have an in-depth look of it.

\section{Evaluation method}
Our evaluation method, described above, consists in comparing the scores of the method implemented with the the \ac{LSB} ones.
They are computed by comparing the original watermarked data with the data recovered after decoding the altered signal. 
To alter the signal, we degrade it using a benchmarking algorithm. We implemented several algorithms based on the \brand{StirMark Benchmark for Audio}. They are described on figure \ref{frameworkclass2}.

\subsection{Algorithm resistance}
Once the algorithm were implemented, we used them to alter the watermarked signal so we could assess the strength of each method. Then we compute a score. 

The process is the following :
\begin{itemize}
 \item We watermark some data into an audio file.
 \item We alter it using a benchmark algorithm.
 \item We decode the signal and recover the watermarked data.
 \item We compare the original data with the recovered one : the bits that do not match are false bits.
\end{itemize}

\subsection{Auditory evaluation}
Another evaluation has also been done : the auditory evaluation. We made about ten persons listen to watermarked signals. 
We then took note of their opinion about the audio file they've listened to : if they heard something different, and if the audio quality was satisfactory. 

The results for both of these evaluations are present on chapter \ref{chap:results}.

\chapter{Organization and project management}
In this chapter, we will study how the project was managed, and the different steps we had to take in order to achieve all our goals.

\section{Technologies and methods}
We saw this project as a way to enact modern software development techniques.
\subsection{Agile methodology}
Even if we didn't explicitely named it, the method used was clearly agile, because every week or every two weeks we saw our clients and showed them our current progress, and we stated the goals for the following weeks.

\subsection{Softwares used}
We also used different software tools to manage the project.

\paragraph{Git and Github} 
They were used to hold the source files and the wiki. We also put our weekly reports here, which allowed the teachers to tell us what we had to improve. \brand{Git} is very useful for advanced softwares, even if we didn't use it to its maximum potential.
As well, we did not use every single feature of \brand{Github}, like bug reports.

\paragraph{Travis CI} 
A big advantage of Github is that it allows continuous integration with other services. We used \brand{Travis CI}. The idea is simple : everytime somebody does a commit, a remote build system fetches it, and tries to build it and runs an user-specified script (in our case, unit tests). If the build fails, or if the software ends in an erroneous state like with a segmentation fault, we get an email, and we can access the build and execution log to see what failed.

This can be used as a testimony to check that the project correctly builds on a minimal system.

\paragraph{Doxygen} was used for the documentation. Mostly everything is documented, the library as well as the graphical interface.

\paragraph{QtCreator} was the chosen development environment. It has a tight integration with \brand{qmake} as a build system, hence the hard dependency on it (which is anyway included with \brand{Qt}, required for the user interface). 

\paragraph{Qt Test Framework}
This test frameworkwas used for the unit testing, in order to provide an easy-to-read output. It can also provide an Xunit XML output which is a standard unit test output, but we did not feel the need for it. 

\paragraph{cppcheck}
It is a static code analysis software which finds bugs unlikely to be found by a compiler, like memory leaks, array bounds overflow... We ran it regularly to check for potential bugs.

\paragraph{valgrind}
A big emphasis was put on the memory cleanliness : as a matter of fact, there is not a single memory leak in the library, thanks to the extensive use of smart pointers (there is not a single free pointer in our classes, they are all embedded in \texttt{std::smart\_ptr} objects).

\paragraph{lcov}
This software was used to test the code coverage of the library during the unit tests, to ensure maximum efficiency for the other tools that we explained before.

The output is present in the folder \texttt{html-coverage}. The tests cover almost $90\%$ of the code (we could not do more because of lack of time).

\section{Schedule}
We will see here the repartition of the work on the project.

An important thing to keep in mind is that the project used an existing framework, which had some bugs : this helped to alleviate much of the required work.

\subsection{Work repartition}
The repartition was as follows : 
\begin{figure}[h!]
\centering
\begin{tabular}{|c|l|}
\hline
Person & Work done \\
\hline
Jean-Michaël Celerier & Work on the framework, project management \\
Abdelhamid Cherif & Evaluation \\
Augustin Chevrier & Compression-expansion method \\
Alban De Martin & \ac{GUI} \\
Quentin Midy & \ac{LSB} Methods \\
Valentin Sarthou & \ac{SSW} Methods \\
Marc Queiros Soares & Compression-expansion method \\
\hline
\end{tabular}
\end{figure}

\subsection{Gantt Diagram}
A Gantt diagram was made to have an assessment and a track to follow. It is viewable on the next page.

\begin{sidewaysfigure}[h!]
\centering
\includegraphics[scale=0.65]{images/gantt.png}
\caption{Approximative Gantt diagram for the project}
\label{gantt}
\end{sidewaysfigure}

\chapter{Implementation}
The following chapter will describe the choices and the inner workings of the project.
The implementation is divided in multiple parts : 

\begin{itemize}
\item An audio manipulation framework.
\item The implementation of many audio and watermarking algorithms, in this framework : the whole makes the library that we call \texttt{libwatermark}.
\item A test application to ensure correctness of the library.
\item A graphical interface that uses the library.
\end{itemize}

All of these parts will be reviewed in the following sections.

\section{Technologies used}
The choice was made really early to use \brand{C++} as a programming language, mostly because of the speed, but also because it is quite cross-platform. We also used many \brand{C++11} features for the sake of learning, and because it sometimes reduces the amount of code needed. However, on \brand{Microsoft Windows} with \brand{Visual Studio}, all the features might not be supported by the compiler.

\subsection{Dependencies}
We tried to use as few dependencies as possible, and we tried to choose the most cross-platform ones. In return, our software should work on every common desktop operating system, and could certainly be embedded in smartphones, etc\dots
\subsubsection{Qt}
The GUI framework. It is not used in the library, but it allowed us to make a fully working GUI in a really short amount of time; also, most of us already had some prior knowledge of it.
\subsubsection{FFTW}
This Fast Fourier Transform library has been used as the main library to perform the Fourier transform needed in some algorithms, because of its efficiency. However, other FFT libraries could be used very easily.
\subsubsection{libsndfile}
An audio file library that allows our software to open \brand{WAV} files and other common filetypes very easily. It is lightweight and efficient. Unfortunately, it doesn't support compressed formats like \brand{MP3} or \brand{OGG Vorbis}. 

\section{Framework}
An audio framework has been designed in order to streamline the audio processing operations and concentrate only on the algorithms.

The other alternatives would have been :
\begin{itemize}
\item To code every algorithm in its own function, with some helper functions to load a file, etc\dots which would have resulted in a lot less code but also a less reusable code.
\item To work in \brand{MATLAB}, which reduces the overhead needed since loading and saving files, as well as transforms, are already implemented.
\end{itemize}

A lot of effort went into making it very generic, fast, and reliable for any kind of audio operations.


\subsection{Operation}
The main idea is : 
\begin{enumerate}
\item To load some audio data.
\item To apply an algorithm to chunks (e.g. 512 samples for instance) of this data.
\item To save the result if needed.
\end{enumerate}

For instance, on a 2048 samples audio file, with a buffer size of 512, there would be four buffers that would each see the algorithm applied to them.

In the current design, there is an input and an output, input and output filters, and the algorithm. For instance, here would be the call order of a generic example.

\begin{enumerate}
\item FFT plotting \textit{calls} FFT \textit{calls} GetNextBuffer
\item Algorithm
\item IFFT \textit{calls} Waveform plotting \textit{calls} WriteNextBuffer
\end{enumerate}

However, this is somewhat clumsy to setup : one must first instantiate the original audio loader I0, then the FFT filter I1 which has to reference I0, then the plotter I2 which has to reference I1, and finally the manager class which has to reference I2, and the same for the output, which leads to bloated code and a logical dissimetry.
So in future project, the framework might be modified to take instead a list of classes which would make our example:

GetNextBuffer $\rightarrow$ FFT $\rightarrow$ FFT plotting $\rightarrow$ Algorithm $\rightarrow$ IFFT $\rightarrow$ Waveform Plotting $\rightarrow$ WriteNextBuffer.

\subsection{Capabilities}
Since an heavy emphasis has been put into having very generic interfaces, we have been able to implement many useful features and algorithms.

The library is entirely templated, which means that most of the classes can be used with fixed or floating point numbers. However, it does not always make sense : for instance, \brand{FFTW} only works in floating point, so using \texttt{short int} as a base sample type would not work.

Multiple channels are supported in a seamless way, and can be interweaved or deinterweaved.

There are interfaces for : 
\begin{itemize}
\item Inputs and outputs \& filters.
\item Transforms.
\item Watermarking \& degradation algorithms.
\item Time management.
\item Watermark embedding.
\item Copying, overlapping and windowing.
\end{itemize}

You can see on figure \ref{frameworkclass} tables that repertories all the implementations of these interfaces.

\begin{figure}[ht!]
\label{frameworkclass}
\centering
\begin{tabular}{|c|c|l|}
\hline
\textbf{Inputs} & \textbf{Outputs} & \textbf{Explanation} \\
\texttt{InputManagerBase} & \texttt{InputManagerBase}  & \\
\hline
FileInput & FileOutput & Wrapper around \brand{libsndfile} \\
BufferInput & BufferOutput & Read and write from a buffer \\
SilenceInput & DummyOutput & Generates silence and writes to nothing \\
\hline
FFTInputProxy & FFTOutputProxy & FFT transform \\
MCLTInputProxy & MCLTOutputProxy & MCLT transform \\
\hline
& GnuplotOutput & Displays sample data in \brand{GNUPlot} \\
& GnuplotFFTOutput & Displays a spectrum in \brand{GNUPlot} \\
\hline
\end{tabular}

\vspace{1em}

\begin{tabular}{|c|c|l|}
\hline
\textbf{Input copy handlers} & \textbf{Output copy handlers} &  \textbf{Explanation} \\
\texttt{InputCopy} & \texttt{OutputCopy} & \\
\hline
InputSimple & OutputSimple & Copies and increment by a buffer size \\
InputOLA & OuputOLA & Copies in an overlap-add fashion \\
InputFilter & OutputFilter & Copies in order to perform convolution \\
\hline
\end{tabular}

\vspace{1em}

\begin{tabular}{|c|l|}
\hline
\textbf{Window} & \textbf{Explanation} \\
\texttt{WindowBase} & \\
\hline
RectWindow & Rectangular window : does nothing \\
BartlettWindow & Bartlett window \\
BlackmanWindow & Blackman window \\
Hann & Hann window \\
Hamming & Hamming window \\ 
\hline
\end{tabular}

\caption{Objects related to input-output}
\end{figure}

\begin{figure}[ht!]
\centering
\begin{tabular}{|c|c|l|}
\hline
\textbf{Watermark encoding} & \textbf{Watermark decoding} & \textbf{Explanation} \\
\texttt{WatermarkBase} & \texttt{WatermarkBase} & \\
\hline
LSBEncode & LSBDecode & LSB algorithm \\
SSWEncode & SSWDecode & SSW algorithm \\
\hline
\end{tabular}

\vspace{1em}

\begin{tabular}{|c|l|}
\hline
\textbf{Audio degradations} & \textbf{Explanation} \\
\texttt{BenchmarkBase} & \\
\hline
AddBrumm & Adds mains-like noise \\
AddWhiteNoise & Adds white noise \\
Amplify & A simple gain \\
Convolution & Performs linear convolution for filtering (e.g. LPF / HPF)\\
Exchange & Exchanges pairs of samples \\
FFTNoise & Adds white noise using a spectral method \\
Invert & Inverts the phase of the audio \\
Smooth & Smoothes the samples \\ 
Stat1 & Also smoothes the samples \\
ZeroCross & Strict noisegate\\
\hline
\end{tabular} 
\caption{Audio algorithms}
\end{figure}

\begin{figure}[ht!]
\centering
\begin{tabular}{|c|l|}
\hline
\textbf{Time management} & \textbf{Explanation} \\
\texttt{TimeAdapter} & \\
\hline
AtTime & Starts the algorithm at a certain buffer \\
Every & Starts the algorithm every $k$ buffer \\
For & Starts the algorithm for $n$ buffers \\
Every\_For & Every $k$ buffers, starts the algorithm and stops it after $n$ buffers\\
\hline
\end{tabular}
\caption{Algorithm triggering}
\end{figure}

\begin{figure}[ht!]
\centering
\begin{tabular}{|c|l|}
\hline
\textbf{Watermark embedding} & \textbf{Explanation} \\
\texttt{WatermarkData} & \\
\hline
SimpleWatermarkData & Puts the watermark a single time \\   
LoopingWatermarkData & Puts the watermark repeatedly \\
\hline
\end{tabular}
\caption{Watermarked data-related objects}
\end{figure}

\newpage

\subsection{Class diagram}

The class diagram only shows child classes when it is relevant; else it would cause too much bloat.

Some important things to note are that :
\begin{itemize}
\item The class diagram shown here is for an instantiation with a watermark algorithm (it is very similar for an evaluation algorithm). 
\item Due to the complexity of the library, we will only show relevant public members : detailed documentation is available thanks to the \brand{Doxygen} documentation.
\item All the classes do reference a \texttt{Parameters} class which holds some common parameters and types used across the library.
\end{itemize}


\begin{figure}[ht!]
\centering
\label{parametersclass}
\begin{tikzpicture}
% % % CLASS DIAGRAM HERE % % %
% Base Manager
\umlclass[template=data\_type]
{Parameters}
{
	size\_type = unsigned long int \\
	complex\_type = complex<data\_type> \\
	samplingRate : size\_type \\
	bufferSize : size\_type \\
	
}
{
	normFactor() : data\_type \\
}
\end{tikzpicture}

\caption{The Parameters class}
\end{figure}

\begin{figure}[ht!]
\centering
\label{fwclassdiagram}
\scalebox{0.5}{
\begin{tikzpicture}
% % % CLASS DIAGRAM HERE % % %
% Base Manager
\umlclass[x=0,y=5]
{WatermarkManager}
{}
{
	prepare() : void \\
	execute() : void \\
}

% WatermarkBase
\umlinterface[x=0,y=2]{WatermarkBase}{}
{
	\umlvirt{operator()() : void} \\
}

\umlinterface[x=0, y=-1]{SpectralWatermarkBase}{}{}

% IO
\umlinterface[x=-6,y=6]{IOManagerBase}{}
{
	channels() : size\_type \\
	frames() : size\_type \\
	v() : vector<vector<data\_type>> \\
}
\umlinterface[x=-8,y=2]{InputManagerBase}{}
{
	getNextBuffer() : Buffer \\
}
\umlinterface[x=-8,y=-1]{OutputManagerBase}{}
{
	writeNextBuffer(Buffer) : void \\
}

% WM data
\umlinterface[x=4,y=2]{WatermarkData}{}
{
	setSize() : void\\
	nextBit() : bool\\
	setNextBit(Bool) : void \\
	isComplete() : bool \\
}

% TA
\umlinterface[x=10,y=2]{TimeAdapter}{}
{
addStartHandler(Function) : void\\
addStopHandler(Function) : void\\
perform() : void \\
}
% Links
\umlinherit{InputManagerBase}{IOManagerBase}
\umlinherit{OutputManagerBase}{IOManagerBase}
\umlinherit{SpectralWatermarkBase}{WatermarkBase}

\umlaggreg{InputManagerBase}{WatermarkManager}
\umlaggreg{OutputManagerBase}{WatermarkManager}
\umlaggreg{WatermarkBase}{WatermarkManager}
\umlaggreg{WatermarkData}{WatermarkManager}
\umlaggreg{TimeAdapter}{WatermarkManager}

\end{tikzpicture}
}
\caption{Class diagram for the framework}
\end{figure}

\subsection{Example of usage}
\section{Algorithms}
\subsection{Watermarking algorithms}
\subsection{Evaluation algorithms}
\subsection{Transforms}
\subsubsection{Fourier transform}
\subsubsection{Modulated complex lapped transform}
\section{Tests}
\section{GUI}
\subsection{Capabilities}

\subsection{Screenshots}
\section{Evaluation}
\chapter{Results}
\section{Evaluation results}
The evaluation has been made for the LSB (the reference), the RLSB and the SSW technique with the Amplify, Convolution and FFTAmplify degradations. The scores (number of false bits) are :

\begin{figure}[h!]
\centering
\begin{tabular}{|c|c|c|c|}
\hline
\backslashbox{Algorithm}{degradation} & Amplify & Convolution & FFTAmplify \\
\hline
LSB & 5 & 10 & 7 \\
\hline
RLSB & 31 & 37 & 7\\
\hline
SSW & 52 & 71 & 52\\
\hline
\end{tabular}
\end{figure}
\section{Auditory results}
The auditory evaluation, made on about ten persons, wasn't really relevant. The watermark made on the audio files were audible, all of the persons reported a difference. Added to this, the audio files watermarked with the LSB method showed lot of noise, the original song wasn't audible. The experiment with the SSW technique were more relevant as the song was still audible.
\chapter*{Conclusion}
This project was very long and more difficult than we thought at first, hence some lacking functionalities and expected features when comparing the actual state with what was expected when reading the Gantt diagram.

However, we feel satisfied with the resulting work, because it is globally stable and as much bug-free as possible, with a convenient user interface and an excellent scalability and extensibility.

We could acquire a lot of skills in the domain of signal processing and understand points that are commonplace in DSP algorithms, and we feel that most of what we have coded could be reused in side applications : for instance, instead of watermarking algorithms, our library could be easily used to implement and evaluate high performance signal encryption, speech recognition, or any other kind of audio algorithms.

Since the library is pure \brand{C++} with very few dependencies, it could also be used in embedded applications. For instance, it has been tested with success as a guitar effect pedal simulator on a \brand{BeagleBoard}.

Finally, we could run some real-life tests and research, and even resort to using other people for evaluation, which was quite interesting even if we did not have the time nor the resources to make it on a big scale.

\printglossaries
\bibliographystyle{apalike}
\bibliography{watermarking}
\end{document}
